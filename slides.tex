\environment s-mercury

\setvariables[metadata][
    title={A dark theme for quick presentations},
    subtitle={Written in \CONTEXT},
    author={André Kalouguine},
    date={\date},
]

\starttext

\startslide[title={Slide structure}][subtitle={A quick intro}]
    A slide using this presentation inherits from the subsection head. 
    
    Thus a presentation can be organized using sections.
\stopslide

\startslide[title={Using features}] [subtitle={Math}]
    This theme allows a user to use math as they would in any other \CONTEXT\ document.
    
    \startformula
        \sum_{k=1}^{+\infty}\frac{1}{k^2}=\frac{\pi^2}{6}
    \stopformula
\stopslide

\startslide[title={Using features}] [subtitle={A list of the main bodyfont components}]
    \showbodyfont
\stopslide

\startslide[title={Using features}] [subtitle={Showing images and graphs}]
    Seeing as the presentation employs a dark color scheme, adapting the images to the background is a task which rests solely on the shoulders of the end user. An example of a nicely integrated picture.
\stopslide

\startslide[title={Using features}] [subtitle={Step by step presentations}]
    \externalfigure[image.pdf][width=15cm]
\stopslide

\startslide[title={Closing remarks}] [subtitle={Printing a bibliography}]
    Printing a bibliography is important at the end of academic talks.
\stopslide


\stoptext
